\documentclass[a4paper, 11pt]{article}
\usepackage[brazil]{babel}
\usepackage[utf8]{inputenc}
\usepackage{blindtext}
\usepackage[top=3cm, left=2.5cm, right=2cm, bottom=2cm]{geometry}
\usepackage{amsthm,amsfonts}
\usepackage{graphicx}
\setlength\parindent{0pt}

\author{Frederico Queiroz}
\title{Capítulo 4 - Engenharia de Requisitos}

\begin{document}
\maketitle

\section{O que são requisitos?}
\begin{itemize}
    \item \textbf{Requisitos} de um sistema são as descrições do que o sistema deve fazer, os serviços que oferece e as restrições a seu funcionamento.
    \item \textbf{Engenharia de Requisitos} é o processo de descobrir, analisar, documentar e verificar esses serviços e restrições. 
\end{itemize}

\section{Visões da Especificação}
\begin{itemize}
    \item Alguns dos problemas que surgem durante o processo de engenharia de requisitos são as falhas em não fazer uma separação entre diferentes níveis de descrição.
    \item Diferentes visões da especificação são úteis, pois comunicam informações sobre o sistema para diferentes tipos de leitor (stakeholders).
\end{itemize}

\subsection{Requisitos de Usuário}
\begin{itemize}
    \item \textbf{Requisitos de usuário} são declarações, em uma linguagem natural com diagramas, de quais serviços o sistema deverá fornecer a seus usuários
    e as restrições com as quais este deve operar.
    \item Os leitores dos requisitos de usuário não costumam se preocupar com a forma como o sistema será implementado.
    \item Descreve as funções e restrições do sistema de forma abstrata (inteligível pelo usuário/cliente).
    \item Ponto de vista das necessidades da empresa cliente. Não indica uma solução, e sim a necessidade.
    \item Escrito em linguagem natural com diagramas simples (Ex. tabelas).
	\item Não deve usar jargão de software, notações estruturadas ou notações formais. 
\end{itemize}

\subsection{Requisitos de Sistema}
\begin{itemize}
    \item \textbf{Requisitos de Sistema} são descrições mais detalhadas das funções, serviços e restrições operacionais do sistema.
    \item Os leitores dos requisitos de sistemas precisam saber mais detalhadamente o que o sistema fará, porque estão interessados em como ele apoiará
    os processos dos negocios ou porque esstão envolvidos na implementação do sistema.
    \item Devem ser padronizados, completos e consistentes. 
    \item O documento de requisitos do sistema (usado pela equipe de desenvolvimento) deve definir exatamente o que deve ser implementado. 
    \item Pode ser parte do contrato entre o comprador do sistema e os desenvolvedores de software.
	\item São usados por engenheiros de software como ponto de partida para o projeto do sistema.
\end{itemize}

\section{Especificação de Requisitos}
\begin{itemize}
    \item \textbf{Especificação de Requisitos} é o proceso de escrever os requisitos de \textbf{usuário} e de \textbf{sistema} em um \textbf{documento de requisitos}.
	\item O documento de requisitos não deve incluir detalhes da arquitetura ou projeto do sistema.
	\item Idealmente, os requisitos devem ser claros, inequívocos, de fácil compreensão, completos e consistentes.
	\item Na prática, isso é dificil de se conseguir, pois os \textit{stakeholders} interpretam os requisitos de maneiras diferentes.
	\item Muitas vezes, notam-se conflitos e inconsistências inerentes aos requisitos.
\end{itemize}

\subsection{Requisitos \textbf{X} Projeto}
\begin{itemize}
	\item Requisitos definem \textbf{o que} o sistema faz. Nem sempre dizem \textbf{como} ele faz (projeto).
	\item Os requisitos do sistema devem descrever apenas o comportamento externo do sistema e suas restrições operacionais.
	\item Eles não devem se preocupar com a forma como o sistema deve ser projetado ou implementado.
	\item No entanto, durante a escrita dos requisitos é praticamente impossível eliminar todas as informações de projeto, pelas seguintes razões:
	\begin{itemize}
		\item Projetar uma arquitetura inicial do sistema pode ajudar a estruturar a especificação de requisitos. Os requisitos são organizados de acordo com 
		os diferentes subsistemas que compõem o sistema.
		\item O sistema deve interoperar com outros sistemas existentes, que restringem o projeto e impõem requisitos sobre o novo sistema.
		\item O uso de uma arquitetura específica pode ser um requisito externo do sistema.
	\end{itemize}
\end{itemize}

\subsection{Especificação em Linguagem Natural}
\begin{itemize}
    \item Não é fácil padronizar os requisitos usando linguagem natural.
    \item Alguns problemas:
    \begin{itemize}
        \item Falta de clareza: A linguagem natural pode ser imprecisa, vaga e ambígua. Seu significado depende do conhecimento do leitor.
        \item Confusão de requisitos: Requisitos podem se misturar com informações do projeto.
        \item Fusão de requisitos: Diversos requisitos expressos juntos.
    \end{itemize}
    \item Diretrizes Gerais de Redação:
    \begin{itemize}
        \item Adotar um formato padrão e usá-lo em todas as definições de requisitos.
        \item Usar a linguagem de forma simples e consistente para distinguir entre requisitos obrigatórios e os desejáveis.
        \begin{itemize}
            \item Requisitos obrigatórios são requisitos que o sistema tem de dar suporte. Geralmente escrito usando-se `deve'.
            \item Requisitos desejáveis não são essenciais. Geralmente escrito usando-se `pode'.
        \end{itemize}
		\item Usar destaque (negrito, itálico, sublinhado) para partes importantes.
		\item Evitar usar jargões de informática, siglas e acrônimos em requisitos de usuário.
    \end{itemize}
\end{itemize}

\subsection{Especificação em Linguagem Estruturada}
\begin{itemize}
	\item A \textbf{Linguagem Natural Estruturada} é um forma restrita de escrever os \textbf{requisitos do sistema}. Todos os requisitos são 
	escritos de em uma forma-padrão.
	\item Notações de linguagem estruturada usam templates para especificar os requisitos do sistema.
	\item Vantagens:
	\begin{itemize}
		\item Mantém a facilidade de expressão e compreensão da linguagem natural.
		\item Garante algum grau de uniformidade na especificação.
		\item Podem ser escritos formulários especiais.
	\end{itemize}
\end{itemize}

\section{Requisitos Funcionais e Não Funcionais}
\begin{itemize}
	\item Os requisitos de software são frequentemente classificados como \textbf{requisitos funcionais} e \textbf{requisitos não funcionais}.
\end{itemize}

\subsection{Requisitos Funcionais}
\begin{itemize}
	\item Descrevem explicitamente as funcionalidades e serviços do sistema.
	\item Documenta:
	\begin{itemize}
		\item Como o sistema deve reagir a entradas específicas.
		\item Como o sistema deve se comportar em determinadas situações.
		\item Em alguns casos, podem explicitar o que o sistema \underline{não} deve fazer.
	\end{itemize}
\end{itemize}

\subsubsection{Atributos dos Requisitos Funcionais}
\begin{itemize}
	\item A especificação dos requisitos funcionais de um sistema deve ser \textbf{completa} e \textbf{consistente}.
	\item \textbf{Completude}: Todos os serviços requeridos pelo usuário devem estar definidos.
	\item \textbf{Consistência}: Os requisitos não devem ter definições contraditórias.
	\item Na prática, é quase impossível atingir completude e consistência dos requisitos.
\end{itemize}

\subsubsection{Exemplos de Requisitos Funcionais}
\begin{itemize}
	\item O usuário \underline{pode} pesquisar todo ou um sub-conjunto do banco de dados.
	\item O sistema \underline{deve} oferecer telas apropriadas para o usuário ler documentos armazenados.
	\item Cada pedido \underline{deve} ser associado a um identificador único (PID), o qual o usuário pode copiar para a área de armazenamento permanente da conta.
\end{itemize}

\subsection{Requisitos Não Funcionais (RNF)}
\begin{itemize}
	\item Definem propriedades e restrições do sistema (Ex. segurança, desempenho, espaço em disco).
	\item Não estão diretamente relacionados com os serviços específicos oferecidos pelo sistema a seus usuários.
	\item Podem estar relacionados às propriedades emergentes do sistema, como confiabilidade, tempo de resposta e ocupação de área.
	\item Requisitos não funcionais são frequentemente mais críticos que requisitos funcionais individuais (Se não satisfaz, o sistema é inútil).
\end{itemize}

\subsubsection{Classificação de Requisitos Não Funcionais}
\begin{itemize}
	\item Os requisitos não funcionais podem ser provenientes das características requeridas para o software (requisitos de produto),
	da organização que desenvolve o software (requisitos organizacionais) ou de fontes externas.
	\item \textbf{Requisitos do Produto}: Especificam ou restringem o comportamento do software (Ex.: desempenho, quantidade de memória, usabilidade, confiabilidade).
	\item \textbf{Requisitos Organizacionais}: Derivados das políticas e procedimentos da organização do cliente e do desenvolvedor 
	(Ex.: linguagem de programação, padrões do cliente).
	\item \textbf{Requisitos Externos}: Derivados do ambiente ou dos fatores externos ao sistema e seu processo de desenvolvimento (Ex.: requisitos legais, éticos).
\end{itemize}

\subsubsection{Exemplos de Requisitos Não Funcionais}
\begin{itemize}
	\item \textbf{Requisitos do Produto}: A interface do usuário deve ser implementada como simples HTML.
	\item \textbf{Requisitos Organizacionais}: Todos os documentos entregues devem seguir o padrão de relatórios XYZ-00.
	\item \textbf{Requisitos Externos}: Informações pessoais dos usuários não podem ser vistas pelos operadores do sistema.
\end{itemize}

% [INSERIR IMAGEM DO DIAGRAMA DE RNF]


\subsubsection{Verificação de Requisitos Não Funcionais}
\begin{itemize}
	\item RNF às vezes são difíceis de serem verificados.
	\item Sempre que possível, os requisitos não funcionais devem ser escritos quantitativamente, para que possam ser objetivamente testados.
	\item Na prática, os clientes de um sistema geralmente consideram difícil traduzir suas metas em requisitos mensuráveis.
\end{itemize}

\subsubsection{Métricas de Requisitos Não Funcionais}
% [INSERIR TABELA 4.1 - Métricas para especificar requisitos não funcionais.]


\subsubsection{Alguns Problemas de Requisitos Não Funcionais}
\begin{itemize}
	\item A especificação quantitativa de requisitos não funcionais é difícil.
	\item Ocorre mistura de requisitos funcionais e não funcionais.
	\item Requisitos não funcionais podem conflitar com outros requisitos (funcionais ou não).
\end{itemize}

\section{Processos de engenharia de requisitos}
\begin{itemize}
    \item Os processos de engenharia de requisitos podem incluir quatro atividades de altos nível.
    \item \textbf{Estudo de viabilidade}: Avaliar se o sistema é útil para a empresa.
    \item \textbf{Elicitação e análise}: Descobrir e levantar os requisitos.
    \item \textbf{Especificação de requisitos}: Descrever os requisitos em uma forma-padrão.
    \item \textbf{Validação dos requisitos}: Verificar se os requisitos realmente definem o sistema que o cliente quer.
\end{itemize}

%\today
\end{document}