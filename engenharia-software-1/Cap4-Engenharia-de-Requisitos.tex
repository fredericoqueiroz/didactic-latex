\documentclass[a4paper, 11pt]{article}
\usepackage[brazil]{babel}
\usepackage[utf8]{inputenc}
\usepackage{blindtext}
\usepackage[top=3cm, left=2.5cm, right=2cm, bottom=2cm]{geometry}
\usepackage{amsthm,amsfonts}
\setlength\parindent{0pt}

\author{Frederico Queiroz}
\title{Resumo Capítulo 4 - Engenharia de Requisitos}

\begin{document}
\maketitle

\section{O que são requisitos?}
\begin{itemize}
    \item \textbf{Requisitos} de um sistema são as descrições do que o sistema deve fazer, os serviços que oferece e as restrições a seu funcionamento.
    \item \textbf{Engenharia de Requisitos} é o processo de descobrir, analisar, documentar e verificar esses serviços e restrições. 
\end{itemize}

\section{Visões da Especificação}
\begin{itemize}
    \item Alguns dos problemas que surgem durante o processo de engenharia de requisitos são as falhas em não fazer uma separação entre diferentes níveis de descrição.
    \item Diferentes visões da especificação são úteis, pois comunicam informações sobre o sistema para diferentes tipos de leitor (stakeholders).
\end{itemize}

\subsection{Requisitos de Usuário}
\begin{itemize}
    \item \textbf{Requisitos de usuário} são declarações, em uma linguagem natural com diagramas, de quais serviços o sistema deverá fornecer a seus usuários
    e as restrições com as quais este deve operar.
    \item Os leitores dos requisitos de usuário não costumam se preocupar com a forma como o sistema será implementado.
    \item Descreve as funções e restrições do sistema de forma abstrata (inteligível pelo usuário/cliente).
    \item Ponto de vista das necessidades da empresa cliente. Não indica uma solução, e sim a necessidade.
    \item Escrito em linguagem natural com diagramas simples (Ex. tabelas).
\end{itemize}
\subsubsection{Linguagem Natural}
\begin{itemize}
    \item Não é fácil padronizar os requisitos usando linguagem natural.
    \item Alguns problemas:
    \begin{itemize}
        \item Falta de clareza: A linguagem natural pode ser imprecisa, vaga e ambígua. Seu significado depende do conhecimento do leitor.
        \item Confusão de requisitos: Requisitos podem se misturar com informações do projeto.
        \item Fusão de requisitos: Diversos requisitos expressos juntos.
    \end{itemize}
    \item Diretrizes Gerais de Redação:
    \begin{itemize}
        \item Adotar um formato padrão e usá-lo em todas as definições de requisitos.
        \item Usar a linguagem de forma simples e consistente para distinguir entre requisitos obrigatórios e os desejáveis.
        \begin{itemize}
            \item Requisitos obrigatórios são requisitos que o sistema tem de dar suporte. Geralmente escrito usando-se `deve'.
            \item Requisitos desejáveis não são essenciais. Geralmente escrito usando-se `pode'.
        \end{itemize}
    \end{itemize}
\end{itemize}

\subsection{Requisitos de Sistema}
\begin{itemize}
    \item \textbf{Requisitos de Sistema} são descrições mais detalhadas das funções, serviços e restrições operacionais do sistema.
    \item Os leitores dos requisitos de sistemas precisam saber mais detalhadamente o que o sistema fará, porque estão interessados em como ele apoiará
    os processos dos negocios ou porque esstão envolvidos na implementação do sistema.
    \item Devem ser padronizados, completos e consistentes. 
    \item O documento de requisitos do sistema (usado pela equipe de desenvolvimento) deve definir exatamente o que deve ser implementado. 
    \item Pode ser parte do contrato entre o comprador do sistema e os desenvolvedores de software.
\end{itemize}

\subsection{Especificação de Requisitos}
\begin{itemize}
    \item \textbf{Especificação de Requisitos} é o proceso de escrever os requisitos de \textbf{usuário} e de \textbf{sistema} em um \textbf{documento de requisitos}.
\end{itemize}

%\today
\end{document}