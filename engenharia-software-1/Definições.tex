\documentclass[a4paper, 11pt]{article}
\usepackage[brazil]{babel}
\usepackage[utf8]{inputenc}
\usepackage{blindtext}
\usepackage[top=3cm, left=2.5cm, right=2cm, bottom=2cm]{geometry}
\usepackage{amsthm,amsfonts}
\setlength\parindent{0pt}

\author{Frederico Queiroz}
\title{Conceitos e Definições}

\begin{document}
\maketitle


%\section{Definições}

\section{Software}
\begin{itemize}
	\item \textbf{Software}: Programa de computador + Documentação associada.
	\item Desafios de Produzir Software: Confiabilidade, Preço e Desempenho, Sistemas Críticos.
	\item O desenvolvimento informal de software não é mais suficiente. Técnincas e métodos são necessários.
	\item Dificuldades em se produzir software: Heterogeneidade, Confiabilidade, Prazo de entrega, Mudança contínua.
\end{itemize}

\section{Engenharia de Software}
\begin{itemize}
	\item \textbf{Engenharia de Software}: É uma disciplina de engenharia relacionada a todos os aspectos de produção de software.
	\item Foco no desenvolvimento de software de \textbf{alta qualidade} dentro de \textbf{custos adequados}. Atender necessidades do cliente.		
	
	\item Diferença entre Engenharia de Software e outras Engenharias:
	\begin{itemize}
		\item Software é desenvolvido, não fabricado.
		\item Software não se desgasta.
		\item Software é geralmente produzido para um cliente específico.
	\end{itemize}
	
	\item Pode ser organizada em camadas com foco em qualidade:
	\begin{itemize}
		\item \textbf{Qualidade}: Atributos de um bom software (Facilidade de manutenção, Confiança, Eficiência, Usabilidade, etc).
		
		\item \textbf{Processos de Software}: Atividades (e seus resultados) para o desenvolvimento de software (O que fazer?).
		\item Atividades principais:
		\begin{itemize}
			\item Especificação de requisitos
			\item Modelagem (Projeto de software)
			\item Implementação
			\item Verificação e Validação
			\item Evolução
		\end{itemize}
		
		\item \textbf{Métodos de Software}: Técnicas para desenvolvimento de software (Como fazer?). Métodos incluem Modelos, Notações, Regras, etc.
		
		\item \textbf{Ferramentas}: Fornecem apoio automatizado (ou semiautomatizado) para o processo e para os métodos (Ex.: ferramentas de modelagem do processo).
	\end{itemize}
\end{itemize}


\section{Processos de Desenvolvimento de Software}
\begin{itemize}
	\item %Slide Parte 3
\end{itemize}


	\item \textbf{}
	\item \textbf{}
	\item \textbf{}
	
	
	\begin{itemize}
		\item
	\end{itemize}


\end{document}