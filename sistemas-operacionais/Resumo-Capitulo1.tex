\documentclass[a4paper, 11pt]{article}
\usepackage[brazil]{babel}
\usepackage[utf8]{inputenc}
\usepackage{blindtext}
\usepackage[top=3cm, left=2.5cm, right=2cm, bottom=2cm]{geometry}
\usepackage{amsthm,amsfonts}
\setlength\parindent{0pt}

\author{Frederico Queiroz}
\title{Resumo Capítulo 1}

\begin{document}
\maketitle

\section{Introdução}
- Um \textbf{sistema opercaional} é um programa que gerencia o hardware do computador.

- Ele também fornece uma base para os programas e aplicativos e atua como intermediário entre o usuário e o hardware do computador.

- Como um SO é grande e complexo, deve ser criado em módulos. %
Cada um desses módulos deve ser uma parcela bem delineada do sistema, com entradas e saídas e funções bem definidas.

\subsection{O que Fazem os Sistemas Operacionais}
- Um sistema de computação pode ser grosseiramente dividido em quatro componentes: %
o \textsl{hardware}, o \textsl{sistema operacional (SO)}, os \textsl{programas aplicativos} e os \textsl{usuários}.

- O sistema operacional controla o hardware e coordena seu uso pelos diversos programas aplicativos de vários usuários.

- O SO proporciona um ambiente no qual outros programas possam utilizar o hardware e desempenhar tarefas úteis. 

\vspace{0.5cm}
Examinaremos o SO a partir de dois ponntos de vista: o do \textbf{usuário} e o do \textbf{sistema}.

\subsubsection{O Ponto de Vista do Usuário}
- Em geral, o sistema é projetado para que um único usuário monopolize os recrusos do PC. %
O objetivo é maximizar o trabalho que o usuário estiver executando.

- Nesse caso, o sistema operacional é projetado principalmente para \textbf{facilidade de uso}, com alguma atenção dada ao desempenho e nenhuma à \textbf{utilização dos recursos}.

- O desempenho também é importante para o usuário, mas esses sistemas são otimizados para a experiêcia de um único usuário, e não para atender vários usuários.

\subsubsection{O Ponto de Vista do Sistema}
- Do ponto de vista do sistema, o SO é o programa mais intimamente envolvido com o hardware. Nesse contexto, podemos considerá-lo como um \textbf{alocador (gerenciador) de recursos}.

- Ao lidar com solicitações de recursos numerosas e possivelmente concorrentes, o SO precisa decidir como alocá-los de maneira justa e eficiente.

- Um sistema operacional é um programa de controle. Um \textbf{programa de controle} gerencia a execução dos programas de usuário para evitar erros e uso impróprio do computador.

- Ele se preocupa principalmente com a operação e controle de dispositivos de I/O.


\subsubsection{Definindo Sistemas Operacionais}
- O objetivo fundamental dos sistemas de computação é executar programas do usuário e facilitar a resolução de problemas. É com esse objetivo que o hardware é construído.

- Esses programas requerem determinadas operações comuns, como as que controlam os dispositivos de I/O, por exemplo.

- As funções comuns de controle e alocação de recurso são então reunidas em um tipo de software: o \textbf{sistema operacional}.

\subsection{Organização do Sistema de Computação}


%\today
\end{document}