\documentclass[a4paper, 10pt]{article}
\usepackage[brazil]{babel}
\usepackage[utf8]{inputenc}
\usepackage{blindtext}
\usepackage[top=3cm, left=2.5cm, right=2cm, bottom=2cm]{geometry}
\usepackage{amsthm,amsfonts}
\setlength\parindent{0pt}

\author{Frederico Queiroz}
\title{Capítulo 1 - Exercícios Sugeridos}

\begin{document}
\maketitle

\section{Exercícios}

\subsection{Quais são as três finalidades principais de um sistema operacional?}
- Gerenciar o uso do hardware do computador.

- Fornecer uma base para que os programas e aplicativos possam utilizar o hardware e desempenhar tarefas.

- Controlar a alocação de recursos e a execução de programas de forma eficiente e justa, evitando erros e uso impróprio do computador por parte do usuário.

\setcounter{subsection}{3}
\subsection{Enfatizamos a necessidade de o sistema operacional usar eficientemente o hardware do computador. Quando é apropriado que o SO ignore esse princípio e %
``desperdice'' recursos? Por que um sistema assim não está na verdade sendo eficiente?}

- Quando o sistema de computação é projetado para que um único usuário monopolize o uso de recursos. Nesse caso, o SO é projetado para propocionar facilidade de uso para %
o usuário, não dando atenção a utilização dos recursos, já que o objetivo é maximizar o desemepenho para melhorar o uso.

\setcounter{subsection}{7}
\subsection{Qual das instruções a seguir deve ser privilegiada?}

a. Configurar o valor do timer.

b. Ler o relógio.

c. Apagar a memória.

d. Emitir uma instrução de execução.

e. Desativar interrupções.

f. Modificar entradas na tabela de status de dispositivos.

g. Passar da modalidade de usuário para a de kernel.

h. Acessar dispositivo de I/O.

\end{document}