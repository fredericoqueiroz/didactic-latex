\documentclass[a4paper, 11pt]{article}
\usepackage[brazil]{babel}
\usepackage[utf8]{inputenc}
\usepackage{blindtext}
\usepackage[top=3cm, left=2.5cm, right=2cm, bottom=2cm]{geometry}
\usepackage{amsthm,amsfonts}
\setlength\parindent{0pt}

\author{Frederico Queiroz}
\title{Conceitos e Definições}

\begin{document}
\maketitle

\section{Estruturas}

\subsection{Sistemas Multiprocessados (multiprocessadores)}
\begin{itemize}
    \item Sistemas com múltiplas CPUs próximas.
    \item Sistemas paralelos ou fortemente acoplados
    \begin{itemize}
        \item Processadores compartilham memória e clock
        \item Comunicação entre CPUs pela memória
    \end{itemize}
    \item Vantagens:
    \begin{itemize}
        \item Aumento de desempenho(throughput)
        \item Economia de escala
        \item Aumento da confiabilidade
    \end{itemize}
\end{itemize}

\subsection{Sistemas Distribuídos}
\begin{itemize}
    \item Sistema fracamanete acoplados.
    \begin{itemize}
        \item Cada processador tem sua memória local.
        \item Comunicação se dá por canais de transmissão.
    \end{itemize}
\end{itemize}

\end{document}