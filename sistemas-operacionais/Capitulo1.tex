\documentclass[a4paper, 12pt]{article}
\usepackage[brazil]{babel}
\usepackage[utf8]{inputenc}
\usepackage{blindtext}
\usepackage[top=3cm, left=2.5cm, right=2cm, bottom=2cm]{geometry}
\usepackage{amsthm,amsfonts}
\usepackage{graphicx,color}
\setlength\parindent{0pt}

\begin{document}

\LARGE{Capitulo 1}

\section{Introdução}

Um \textbf{sistema opercaional} é um programa que gerencia o hardware do computador. Ele também fornece uma base para os programas e aplicativos e atua como intermediário entre o usuário e o hardware do computador. \par

Como um SO é grande e complexo, deve ser criado em módulos. Cada um desses módulos deve ser uma parcela bem delineada do sistema, com entradasm saídas e funções bem definidas. \par


\subsection{O que Fazem os Sistemas Operacionais}

Um sistema de computação pode ser grosseiramente dividido em quatro componentes: o hardware, o sistema operacional, os programas aplicativos e os usuários. \\

O sistema operacional controla o hardware e coordena seu uso pelos diversos programas aplicativos de vários usuários. \\

\subsubsection{O Ponto de Vista do Usuário}




\end{document}